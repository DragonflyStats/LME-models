


 I would like to switch from nlme to lme4 and try to translate some of my
 models that worked fine with lme.
 I have problems with the pdMat classes.

 Below a toy dataset with a fixed effect F and a random effect R. I gave
 also 2 similar lme models.

 The one containing pdLogChol (lme1) is easy to translate (as it is an
 explicit notation of the default model)


 The more parsimonious model with pdDiag replacing pdLogChol I cannot
 reproduce with lmer. The obvious choice for me would be my model lmer2,
 but this is yielding different result.

 %===============================================================================%


 #dataset from McLean, Sanders and Stroup
 F <- factor(c(1,1,1,1,1,1,2,2,2,2,2,2))
 R <- factor(c(1,1,2,2,3,3,1,1,2,2,3,3))
 s <-
 c(51.43,51.28,50.93,50.75,50.47,50.83,51.91,52.43,52.26,52.33,51.58,51.23)
 DS <- data.frame(F,R,s)
 DS$F <- as.factor(DS$F)
 DS$R <- as.factor(DS$R)

 library(nlme)
 lme1 <- lme(data = DS,s ~ F,random = list(R = pdLogChol(~F)))
 lme2 <- lme(data = DS,s ~ F,random = list(R = pdDiag(~F)))
 summary(lme1)
 summary(lme2)

 library(lme4)
 lmer1 <- lmer(data = DS,s ~ F + (F|R))
 lmer2 <- lmer(data = DS,s ~ F + (1|R) + (1|F))
 summary(lmer1)
 summary(lmer2)

You need to generate the two columns of the indicator matrix for F as
separate model matrices.  This will involve some awkward construction
like

 (fm1 <- lmer(s ~ F +(F|R), DS))
Linear mixed-effects model fit by REML
Formula: s ~ F + (F | R)
   Data: DS
      AIC      BIC    logLik MLdeviance REMLdeviance
 20.69259 23.60203 -4.346297    6.03915     8.692593
Random effects:
 Groups   Name        Variance Std.Dev. Corr
 R        (Intercept) 0.108796 0.32984
          F2          0.102008 0.31939  -0.014
 Residual             0.048525 0.22028
# of obs: 12, groups: R, 3

Fixed effects:
            Estimate Std. Error DF  t value  Pr(|t|)
(Intercept)  50.9483     0.2106 10 241.9188 < 2.2e-16
F2            1.0083     0.2240 10   4.5014  0.001141
 (fm2 <- lmer(s ~ F + (0+as.numeric(F==1)|R) + (0+as.numeric(F==2)|R), DS))
Linear mixed-effects model fit by REML
Formula: s ~ F + (0 + as.numeric(F == 1) | R) + (0 + as.numeric(F ==
2) |      R)
   Data: DS
      AIC      BIC    logLik MLdeviance REMLdeviance
 19.62621 22.05075 -4.813107   7.439584     9.626215
Random effects:
 Groups   Name               Variance Std.Dev.
 R        as.numeric(F == 1) 0.108796 0.32984
 R        as.numeric(F == 2) 0.207896 0.45596
 Residual                    0.048525 0.22028
# of obs: 12, groups: R, 3; R, 3

Fixed effects:
            Estimate Std. Error DF  t value Pr(|t|)
(Intercept) 50.94833    0.21060 10 241.9188  < 2e-16
F2           1.00833    0.34891 10   2.8899  0.01611

There are probably more elegant ways of doing this but I don't think
there is any really clean way.  If you want to assume that all the
variances are equal then you can estimate the model using an
interaction.

 (fm3 <- lmer(s ~ F + (1|F:R), DS))
Linear mixed-effects model fit by REML
Formula: s ~ F + (1 | F:R)
   Data: DS
      AIC     BIC    logLik MLdeviance REMLdeviance
 17.77917 19.7188 -4.889587   7.669022     9.779175
Random effects:
 Groups   Name        Variance Std.Dev.
 F:R      (Intercept) 0.158346 0.39793
 Residual             0.048525 0.22028
# of obs: 12, groups: F:R, 6

Fixed effects:
            Estimate Std. Error DF  t value Pr(|t|)
(Intercept) 50.94833    0.24672 10 206.5049  < 2e-16
F2           1.00833    0.34891 10   2.8899  0.01611


