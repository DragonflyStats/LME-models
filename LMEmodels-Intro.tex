A mixed model is a statistical model containing both fixed effects and random effects. These models are useful in a wide variety of disciplines in the physical, biological and social sciences. They are particularly useful in settings where repeated measurements are made on the same statistical units (longitudinal study), or where measurements are made on clusters of related statistical units. Because of their advantage in dealing with missing values, mixed effects models are often preferred over more traditional approaches such as repeated measures ANOVA.


History and current status[edit]
Ronald Fisher introduced random effects models to study the correlations of trait values between relatives.[1] In the 1950s, Charles Roy Henderson provided best linear unbiased estimates (BLUE) of fixed effects and best linear unbiased predictions (BLUP) of random effects.[2][3][4][5] Subsequently, mixed modeling has become a major area of statistical research, including work on computation of maximum likelihood estimates, non-linear mixed effect models, missing data in mixed effects models, and Bayesian estimation of mixed effects models. Mixed models are applied in many disciplines where multiple correlated measurements are made on each unit of interest. They are prominently used in research involving human and animal subjects in fields ranging from genetics to marketing, and have also been used in industrial statistics.[citation needed]

Definition[edit]
In matrix notation a mixed model can be represented as

\boldsymbol{y} = X \boldsymbol{\beta} + Z \boldsymbol{u} + \boldsymbol{\epsilon}
where

\boldsymbol{y} is a known vector of observations, with mean E(\boldsymbol{y}) = X \boldsymbol{\beta};
\boldsymbol{\beta} is an unknown vector of fixed effects;
\boldsymbol{u} is an unknown vector of random effects, with mean E(\boldsymbol{u})=\boldsymbol{0} and variance-covariance matrix \operatorname{var}(\boldsymbol{u})=G;
\boldsymbol{\epsilon} is an unknown vector of random errors, with mean E(\boldsymbol{\epsilon})=\boldsymbol{0} and variance \operatorname{var}(\boldsymbol{\epsilon})=R;
X and Z are known design matrices relating the observations \boldsymbol{y} to \boldsymbol{\beta} and \boldsymbol{u}, respectively.
