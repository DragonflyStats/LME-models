%%%%%%%%%%%%%%%%%%%%%%%%%%%%%%%%%%%%%%%%%%%%%%%%%%%%%%%%%%%%%%%%%%%%%%%%%%%%%%%%%%%%%%%%%%%%%%%%%%%%%%%%%%%%%%%%%%
\section{Grouped Data Sets}
In modern statistical analysis , data sets have very complex
structures, such as  clustered data, repeated data and
hierarchical data (henceforth referred to as grouped data).

Repeated data considers various observations periodically taken
from the same subjects. `Before and after' measurements, as used
in paired t tests, are a well known example of repeated
measurements. Clustered data is simply the grouping of
observations according to common characteristics. For example, an
study of pupils of a school would account for the fact that they
are grouped according to their classes.

Hierarchical structures organize data into a tree-like structure,
i.e. groups within groups. Using the previous example, the pupils
would be categorized according to their years (i.e the parent
group) and then their classes (i.e the child group). This can be
extended again to multiple schools, where each school would be the
parent group of each year.

An important feature of such data sets is that there is
correlation between observations within each of the groups
\citep{Faraway}. Observations in different groups may be
independent, but any assumption that these observations within the
same group are independent is inappropriate . Consequently
\citet{Demi} states that there is two sources of variations to be
considered, `within groups' and `between groups'.

