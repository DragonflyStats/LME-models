% 11.1
Fitting LME models with fixed effects and correlated errors using the
\texttt{gls} function.

An important aspect of a model is the correlation function, which is used to
take into account correlation of observations belonging to the same group.

%-------------------------------------------------------------%

% 11.2 Correlation-Structure Representation : the \texttt{corStruct} Class

An important component needed in the context of LME models is the
correlation structure for residual errors.

The \texttt{corStruct} class is implemented using the \textbf{nlme} package.

%--------------%
% Page 198
% 11.2 Correlation Structure Constructor Functions

Initialization of objects of class corStruct

%-------------------------------------------------------------%
% Page 199-202
% 11.3 Inspecting an Modifying objects of class "corStruct"
%      11.3.1  Coefficients of correlation stuctures
%      11.3.2  Semvariogram
%      11.3.3  The "corMatrix" Function
%-------------------------------------------------------------%

% 11.3.2 Semi-variogram

\texttt{Variogram} can be applied to objects inherited from the
\texttt{corSpatial} class.


%-------------------------------------------------------------%
% Page 202- 209
% 11.4 Illustration of Correlation Structure
%      11.4.1 Compound Symmetry :  The "corCompSymm" class
%      11.4.2 Autoregressive Structure of Order 1 :  The "corAR1" class 
%      11.4.3 Exponential Structure : The "corExp" class
%-------------------------------------------------------------%

\section{Using the \texttt{gls()} function}  %11.5 page  209

%-------------------------------------------------------------%

\section{Extarcting Information from a Model-Fit object of class gls}  %11.6 page  210
