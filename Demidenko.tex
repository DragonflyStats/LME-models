\documentclass[12pt, a4paper]{article}
\usepackage{natbib}
\usepackage{vmargin}
\usepackage{graphicx}
\usepackage{epsfig}
\usepackage{subfigure}
%\usepackage{amscd}
\usepackage{amssymb}
\usepackage{amsbsy}
\usepackage{amsthm, amsmath}
%\usepackage[dvips]{graphicx}
\bibliographystyle{chicago}
\renewcommand{\baselinestretch}{1.8}
% left top textwidth textheight headheight % headsep footheight footskip
\setmargins{3.0cm}{2.5cm}{15.5 cm}{23.5cm}{0.5cm}{0cm}{1cm}{1cm}
\pagenumbering{arabic}


\begin{document}
\author{Kevin O'Brien}
\title{Mixed Effects Models (Demidenko)}
\date{\today}
\maketitle
\tableofcontents \setcounter{tocdepth}{2}

\newpage
\section{REML [pg 58]}
Demidenko[pg58]

\newpage
\section{Likelihood Ratio Tests[pg 135]}
Statistical testing of the presence of random effects.
\newpage
\section{Errors in Variables[pg 118]}
\begin{itemize}
\item[Identifiability]
Conditions user which LME model is identifiable are set out in section 2.2. Identifiability is a necessary, but not sufficient, property for the adequacy of any statistical model.
\item[Confidence Intervals] Construction and statistical testings on beta parameters in the LME model are based on the asymptotic covariance matrix [given as 3.9].
    \begin{itemize}
    \item Remark upon the association with Wald testing (Wald CIs).
    \item Use of profile-likelihood (PL) intervals by contrast to Wald CIs in non-linear statistical models.
    \end{itemize}
\end{itemize}

\addcontentsline{toc}{section}{Bibliography}

\bibliography{2012bib}

\end{document} 