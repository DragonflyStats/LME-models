\documentclass[12pt, a4paper]{article}
\usepackage{epsfig}
\usepackage{subfigure}
%\usepackage{amscd}
\usepackage{amssymb}
\usepackage{amsbsy}
\usepackage{amsthm}
%\usepackage[dvips]{graphicx}
\usepackage{natbib}
\bibliographystyle{chicago}
\usepackage{vmargin}
% left top textwidth textheight headheight
% headsep footheight footskip
\setmargins{3.0cm}{2.5cm}{15.5 cm}{22cm}{0.5cm}{0cm}{1cm}{1cm}
\renewcommand{\baselinestretch}{1.5}
\pagenumbering{arabic}
\theoremstyle{plain}
\newtheorem{theorem}{Theorem}[section]
\newtheorem{corollary}[theorem]{Corollary}
\newtheorem{ill}[theorem]{Example}
\newtheorem{lemma}[theorem]{Lemma}
\newtheorem{proposition}[theorem]{Proposition}
\newtheorem{conjecture}[theorem]{Conjecture}
\newtheorem{axiom}{Axiom}
\theoremstyle{definition}
\newtheorem{definition}{Definition}[section]
\newtheorem{notation}{Notation}
\theoremstyle{remark}
\newtheorem{remark}{Remark}[section]
\newtheorem{example}{Example}[section]
\renewcommand{\thenotation}{}
\renewcommand{\thetable}{\thesection.\arabic{table}}
\renewcommand{\thefigure}{\thesection.\arabic{figure}}
\title{Research notes: linear mixed effects models}
\author{ } \date{ }


\begin{document}
\author{Kevin O'Brien}
\title{LME Estimation and Algorithms}

\tableofcontents \setcounter{tocdepth}{2}
%------------------------------------------------- MLE Algorithms %



\subsubsection*{Estimation of the fixed parameters}

The vector $y$ has marginal density $y \sim \mathrm{N}(X \beta,V),$ where $V = \Sigma + ZDZ^\prime$ is specified through the variance component parameters $\theta.$ The log-likelihood of the fixed parameters $(\beta, \theta)$ is
\begin{equation}
	\ell (\beta, \theta|y) =
	-\frac{1}{2} \log |V| -\frac{1}{2}(y -
	X \beta)'V^{-1}(y -
	X \beta), \label{Likelihood:MarginalModel}
\end{equation}
and for fixed $\theta$ the estimate $\hat{\beta}$ of $\beta$ is obtained as the solution of
\begin{equation}
	(X^\prime V^{-1}X) {\beta} = X^\prime V^{-1}y.
	\label{mle:beta:hat}
\end{equation} 

Maximum likelihood and restricted maximum likelihood have become the most common strategies for estimating the variance component parameter $\theta.$ Substituting $\hat{\beta}$ from (\ref{mle:beta:hat}) into $\ell(\beta, \theta|y)$ from (\ref{Likelihood:MarginalModel}) returns the \emph{profile} log-likelihood
\begin{eqnarray*}
	\ell_P(\theta \mid y) &=& \ell(\hat{\beta}, \theta \mid y) \\ 
	&=& -\frac{1}{2} \log |V| -\frac{1}{2}(y - X \hat{\beta})'V^{-1}(y - X \hat{\beta})
\end{eqnarray*}
of the variance parameter $\theta.$ Estimates of the parameters $\theta$ specifying $V$ can be found by maximizing $\ell_P(\theta \mid y)$ over $\theta.$ In practice the \emph{restricted} log-likelihood
\[
\ell_R(\theta \mid y) =
\ell_P(\theta \mid y) -\frac{1}{2} \log |X^\prime VX |
\]
is preferred. This approach is based on maximizing the likelihood of linear combinations of $y$ that do not depend on $\beta,$ and in this way takes into account the estimation of $\beta.$


\subsubsection*{Estimation of the random effects}

The established approach for estimating the random effects is to use the best linear predictor of $b$ from $y,$ which for a given $\beta$ equals $DZ^\prime V^{-1}(y - X \beta).$ In practice $\beta$ is replaced by an estimator such as $\hat{\beta}$ from (\ref{mle:beta:hat}) so that $\hat{b} = DZ^\prime V^{-1}(y - X \hat{\beta}).$ Pre-multiplying by the appropriate matrices it is straightforward to show that these estimates $\hat{\beta}$ and $\hat{b}$ satisfy the equations in (\ref{Henderson:Equations}).


\addcontentsline{toc}{section}{Bibliography}
\bibliography{2012bib}
\end{document} 
