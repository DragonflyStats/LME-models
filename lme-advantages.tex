\documentclass[main.tex]{subfiles}
\begin{document}

\subsection{Advantages of Mixed Models}
\begin{itemize}
\item \textbf{cite:BrownPrescott} discusses the  following advantages of using
mixed effects models. In the case of repeated measurements , it is
appropriate to take account of the correlation of each group of
observations. 
\item 
Mixed models lead to more appropriate estimates and
standard errors for fixed effects, particularly in the case of
repeated measures. Analysis using a mixed model is more
appropriate for inference on a hierarchical data. In the case of
unbalanced data, mixed models are more appropriate than other
methodologies.
\item 
\textbf{cite:Demidenko} comments that mixed models are the correct approach
for dealing with grouped data. The use of linear mixed effects
models has advanced greatly with increased usage of statistical
software. 
\item 
This author also notes that mixed models are a hybrid of
bayesian and frequentist methodologies and that mixed model
approaches are more flexible than Bayesian.
\end{itemize}

\subsubsection{Unbalanced Data} 
\begin{itemize}
\item Unbalanced data refers to situations where these groups are
of different sizes. Mixed Effects Models are suitable for studying
unbalanced data sets. 
\item The variance components of random effects
for these set can not be derived using alternative methods such as
ANOVA.
\end{itemize}
\end{document}