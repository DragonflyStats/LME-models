
\subsection{Variance-Covariance Structures}

\subsubsection{Independence}

As though analyzed using between subjects analysis.
\[
\left(
\begin{array}{c c c}
  \psi^2 & 0 & 0   \\
  0 & \psi^2 & 0   \\
  0 & 0 & \psi^2   \\
\end{array}%
\right)
\]


\subsubsection{Compound Symmetry}

Assumes that the variance-covariance structure has a single variance (represented by $\psi^2$)
for all 3 of the time points and a single covariance (represented by $\psi_{ij}$) for each of the pairs of trials.

\[
\left(%
\begin{array}{c c c}
  \psi^2 &  \psi_{12} & \psi_{13}   \\
  \psi_{21} & \psi^2 & \psi_{23}   \\
  \psi_{31} & \psi_{32} & \psi^2   \\
\end{array}%
\right)
\]


\subsubsection{Unstructured}

Assumes that each variance and covariance is unique.
Each trial has its own variance (e.g. s12 is the variance of trial 1)
and each pair of trials has its own covariance (e.g. s21 is the covariance of trial 1 and trial2).
This structure is illustrated by the half matrix below.


\subsubsection{Autoregressive}

Another common covariance structure which is frequently observed
in repeated measures data is an autoregressive structure,
which recognizes that observations which are more proximate
are more correlated than measures that are more distant.
