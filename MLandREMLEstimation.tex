\subsection{Likelihood and estimation}

Likelihood is the hypothetical probability that an event that has
already occurred would yield a specific outcome. Likelihood
differs from probability in that probability refers to future
occurrences, while likelihood refers to past known outcomes.

The likelihood function is a fundamental concept in statistical
inference. It indicates how likely a particular population is to
produce an observed sample. The set of values that maximize the
likelihood function are considered to be optimal, and are used as
the estimates of the parameters.

Maximum likelihood (ML) estimation is a method of obtaining
parameter estimates by optimizing the likelihood function. The
likelihood function is constructed as a function of the parameters
in the specified model.

Restricted maximum likelihood (REML) is an alternative methods of
computing parameter estimated. REML is often preferred to ML
because it produces unbiased estimates of covariance parameters by
taking into account the loss of degrees of freedom that results
from estimating the fixed effects in $\boldsymbol{\beta}$.

REML estimation reduces the bias in the variance component, and also handles high correlations
more effectively, and is less sensitive to outliers than ML.  The problem with REML for model building is that the "likelihoods" obtained for different fixed effects are not comparable. Hence it is not valid to compare models
with different fixed effects using a likelihood ratio test or AIC when REML is used to
estimate the model. Therefore models derived using ML must be used instead.
\newpage



\newpage
Assuming a statistical model $f_{\theta}(y)$ parameterized by a fixed and unknown set of parameters $\theta$, the likelihood $L(\theta)$ is the probability of the observed data $y$ considered as a function of $\theta$ \citep{youngjo}.
