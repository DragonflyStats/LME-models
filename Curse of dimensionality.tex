Curse of dimensionality

The curse of dimensionality refers to various phenomena that arise when analyzing and organizing data in high-dimensional spaces (often with hundreds or thousands of dimensions) that do not occur in low-dimensional settings such as the three-dimensional physical space of everyday experience.
There are multiple phenomena referred to by this name in domains such as numerical analysis, sampling, combinatorics, machine learning, data mining and databases. The common theme of these problems is that when the dimensionality increases, the volume of the space increases so fast that the available data become sparse. This sparsity is problematic for any method that requires statistical significance. In order to obtain a statistically sound and reliable result, the amount of data needed to support the result often grows exponentially with the dimensionality. Also organizing and searching data often relies on detecting areas where objects form groups with similar properties; in high dimensional data however all objects appear to be sparse and dissimilar in many ways which prevents common data organization strategies from being efficient.
The term curse of dimensionality was coined by Richard E. Bellman when considering problems in dynamic optimization.[1][2]
